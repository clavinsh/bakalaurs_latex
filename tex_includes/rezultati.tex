\begin{center}
    \chapter{Rezultāti}
\end{center}

Praktiskai platformu salīdzināšanai tika izstrādāts etalonuzdevumu ietvars, 
lai šie uzdevumi būtu pēc iespējas viegli atkārtojami citos uzstādījumos, tas
ir, ar citu aparatūru. No 2 definētajiem uzdevumiem un 3 pētamajām GPU
programmēšanas platformām tika izveidoti 8 Docker konteineri ar komandrindas
programmām un praktiski notestēti 6 uz Nvidia RTX 3060 Laptop videokartes:
\begin{itemize}
    \setlength\itemsep{0pt}
    \item golcuda - "Dzīves spēles" uzdevuma konteineris uz CUDA,
    \item golcl - "Dzīves spēles" uzdevuma konteineris uz OpenCL,
    \item golhip:cuda - "Dzīves spēles" uzdevuma konteineris uz ROCm HIP ar CUDA aizmuguri,
    \item golhip:rocm - "Dzīves spēles" uzdevuma konteineris uz ROCm HIP ar ROCm aizmuguri,
    \item sha256cuda - paroļu atgūšanas uzdevuma konteineris uz CUDA,
    \item sha256cl - paroļu atgūšanas uzdevuma konteineris uz OpenCL,
    \item sha256hip:cuda - paroļu atgūšanas uzdevuma konteineris uz ROCm HIP ar CUDA aizmuguri,
    \item sha256hip:rocm - paroļu atgūšanas uzdevuma konteineris uz ROCm HIP ar ROCm aizmuguri.
\end{itemize}

Konteineri tika darbināti automatizēti ar Python skriptiem, kuri sagatavoja
programmām ievaddatus, dažādās to darbināšanas konfigurācijās, visās
permutācijās. Dažādos griezumos apskatīti iegūtie dati no žurnālfailiem, un
kopsummā apstrādāti un analizēti 810 žurnālfaili par CUDA, HIP, OpenCL
dažādajiem GPGPU notikumu izpildes ilgumiem.

Noskaidrots, ka paroļu atguvēja uzdevumā OpenCL darbojās par 30,18\% lēnāk nekā
CUDA, bet ROCm HIP platforma par 2,25\% lēnāk. "Dzīves spēles" uzdevumā
rezultāti sarindojās tieši tāpat, lai gan OpenCL starpība nebija tik liela kā
pirmajā uzdevumā, - OpenCL lēnāks par 7,71\%, bet ROCm HIP par 2,47\%.

Aptuveni šāds sniegums bija gaidāms, jo CUDA un ROCm ir gandrīz identiskas
platformas, turpretī OpenCL konveijerapstrādes modelis, SPIR-V starp-posmu
valodas tulkošana un citas ieviestās abstrakcijas rada papildus virsdarbi. ROCm
virsdarbe ir minimāla, jo ar Nvidia videokartēm tiek atbalstīta CUDA
kompilatora aizmugursistēma HIP rakstītām programmām. Tāpēc praktiski
visos rādītājos CUDA bija ātrākā, ROCm HIP bija otrajā vietā, OpenCL - trešajā.

Etalonuzdevumu veikšana pilnīgakiem datiem būtu veicama uz vairākām videokartēm
- kā minimums gan Nvidia, gan AMD. Līdzīgi pētījumi ar vairākām GPU salīdzina
GPGPU programmu veiktspēju starp CUDA un OpenCL, un nonāk pie līdzigiem
secinājumiem -  fakta, ka CUDA darbojas par 30\%, 3.8\% - 5.4\% ātrāk nekā
OpenCL \cite{6047190, memeti2017benchmarkingopenclopenaccopenmp}.

Izmantojamo uzdevumu skaits etalonuzdevumos būtu jāpalielina, norādot konkrētas
praktiskas GPGPU izmantošanas jomas, piemēram, mākslīgā intelekta  trenēšana.
Esošie divi uzdevumi var nebūt pietiekoši konstruktīviem secinājumiem,  kurus
varētu vispārināt vai attiecināt kādām GPGPU programmu jomām. Citos pētījumos
izmantoti, piemēram, specifiski mākslīgā intelekta modeļi
\cite{10.1145/3529538.3529980} vai attēlu apstrādes filtri
\cite{memeti2017benchmarkingopenclopenaccopenmp}, bet fokuss pārsvarā ir vai nu
tikai uz CUDA vai salīdzinājums starp CUDA un OpenCL, bet ne ROCm.

Visās trijās platformās abos uzdevumos bija manāmas rimstošas svārstības
kodolu izpildes laikos, bet papildus tika novērotas konsekventas variācijas un
nestabilitāte kodolu izpildē ar OpenCL un ROCm platformām, kur šīs svārstības
aptuveni 80\% gadījumu nerimst. Pats periods arī nav vienmērīgs, un visos
gadījumos vērojama lielāka nosliece uz mazākām vērtībām. Tāpēc, apskatot
kodolu izpildes laikus kā histogrammas pēc biežumiem, CUDA risinājums veido
vienu pīķi, bet OpenCL un ROCm - divus.

Nerimstošās svārstības visdrīzāk skaidrojamas ar GPU dinamisko sprieguma un
frekvences mērogošanu (DVFS). Pieejamā literatūra par šo tēma ir ierobežota, jo
DVFS algoritmi, aparatūra ir GPU izstrādātāju tirdzniecības noslēpumi.
Pieejamās publikācijas koncentrējas uz DVFS ietekmi uz patērēto enerģiju
\cite{gpu-dvfs, MEI201789, GUERREIRO201993}. Tālāk šajā tēmā būtu pētāma ilgstoši ejošu programmu un/vai
kodolu ietekme no dažādiem DVFS iestatījumiem apskatītajās platformās.

Bet šīs svārstības nav salīdzinoši lielas - ar amplitūdu no 0,0212\si{\ms} līdz
0,1147\si{\ms} kodoliem, kuru vidējais izpildes laiks ir no 0,6446\si{\ms} līdz
2.001\si{\ms}.

Līdzīgi tika apskatīta GPU atmiņas izveide un datu pārsūtīšana, bet šajos
mērījumos iepriekšminētās svārstības netika novērotas nevienā platformā. Datu
buferi \(2^{20}\) parolēm, kas aizņem aptuveni 33.55MiB, tiek nosūtīti vidēji
no 2.7472\si{\ms} līdz 6.4135\si{\ms} ar amplitūdu no 0.03764\si{\ms}  līdz
0.0558\si{\ms}.

Pieejamā akadēmiskā literatūra, pētījumi, kuri salīdzina visas trīs platformas
ir maz \cite{google_scholar, 10.1145/3529538.3529980}, pārsvarā tās tiek
apskatītas individuāli konkrētos pielietojumos (visvairāk CUDA), kas ir
loģiski, jo, kā tika noskaidrots šajā bakalaura darbā, lai gan visas trīs
platformas paradzētas un izstrādātas augstas veiktspējas plašlietojuma GPU
programmēšanas uzdevumiem, reālā tirgus situācija ar Nvidia pārsvaru krietni
novirza GPGPU programmētājus lietot un izstrādāt risinājumus CUDA vidē
\cite{kursa-darbs}. Individuāli CUDA arī ir pārsvarā, jo, ja, piemēram,
atsevišķi meklē resursus LU "PRIMO" iMākonī, tad pieejamie rezultāti darba
izstrādes brīdī (2025. gada maijā) par ROCm ir 400, OpenCL - 2093, bet CUDA -
11117 \cite{primo}.

Būtu vērtīgi veikt neatkarīgus pētījumus šo platformu salīdzināšanā, lai tālāk
izprastu to stiprās un vājās puses. Papildus uzmanība jāvelta AMD izstrādatajam
ROCm risinājumam, par kuru trūkst pētījumi, eksperimenti un programmatūra, kura
izmanto šo CUDA savietojamo un veiktspējā līdzīgo platformu.


