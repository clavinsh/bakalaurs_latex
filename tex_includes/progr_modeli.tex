\begin{center}
    \chapter{Platformu programmēšanas modeļi}
\end{center}
GPU ir dizainēti ar domu izpildīt tūkstošiem pavedienus (vienādus sarakstus ar
procesorā izpildāmām darbībām) paralēli. Atšķirībā no CPU, kur viens pavediens
šīs darbības izpildās ātrāk, GPU kopumā dos lielāku caurlaidību.

Arhitektūras līmenī GPGPU kodoli satur SIMD (\textit{Single Instruction,
Multiple Data}) elementus, kuri spējīgi veikt paralēlu datu apstrādi ar vienu
instrukciju, katrs elements strādājot ar saviem teorētiski patvaļīgiem datiem
(atmiņas adresēm).

SIMD nav pieejami tiešā veidā caur instrukciju kopu, programmētājam ir
jādarbojas ar pavedienu saskarnes SIMT (\textit{Single Instruction, Multiple
Threads}) darbību izpildes modeli. \cite{GPGPU_gramata} Arhitektūrā tie ir
ieviesti kā GPGPU kodoli un satur SIMT elementus. Līdz ar to SIMT kodols tiek
saukt par priekšgalu (\textit{front-end}) un SIMD par aizmugursistēmu
(\textit{back-end}).

SIMT abstrakcija ļauj programmētājam neuztraukties par pavedienu
implementācijas detaļām, un tos var uzskatīt par pilnībā neatkarīgiem - katrs
pavediens izpildās paralēli un ir spējīgs izpildīt patvaļīgu instrukciju
sarakstu.

Rezultātā programmētājs uz GPU izpildāmo kodu, kuru nodod noteiktam skaitam
pavedienu, saucamu par  GPGPU izpildāmo kodolu (\textit{device kernel}), ir
spējīgs definēt kā MIMD (\textit{Multiple Instructions, Multiple Data}) modelim
līdzīgu kodu, tāpēc funkciju definēšana ir praktiski identiska tam kā to darītu
priekš CPU, ar noteiktiem specifiskiem saskarnes API izsaukumiem un kontekstu,
ka šis pavediens izpildās paralēli. \cite{kursa-darbs}

