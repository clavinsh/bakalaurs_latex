\begin{center}
    \chapter{Secinājumi}
\end{center}

Bakalaura darba mērķis bija teorētiski salīdzināt CUDA, ROCm un OpenCL
programmēšanas modeļus, valodas un piedāvāto bibliotēku funkcijas, lielākā
uzmanība pievērsta praktiskajam - izmērīt un analizēt platformu veiktspēju pēc
dažādiem to parametriem, nonākot pie secinājuma kādiem darbiem kura platforma
piemērotāka.

Ņemot vērā, ka visas trīs platformas ir domātas paralēlas izpildes iekārtām,
tad augstā abstrakcijas līmenī tās ir diezgan līdzīgas - programmas CPU puse
pārvalda  mērķa iekārtas atmiņas iedalīšanu, datu izsūtīšanu un saņemšanu,
kodolu darbināšanu. Atšķirības starp platformām ir detaļās - programmēšanas
modelis, valodu specifikācija, pieejamās C/C++ programmēšanas  saskarņu
funkcijas un to nianses, atbalstītās iekārtas.

Praktiskai rīku salīdzināšanai, lai iegūtu uzticamus skaitliskus mērījumus, 
bija nepieciešamība iepazīties ar labās prakses etalonuzdevumu izveidi.
Rezultātā izveidotais ietvars tika lietots etalonuzdevumu skriptu, kā arī pašu
mērāmo programmu izstrādei.

Pēc iegūtajiem etalonuzdevumu datiem secināts, ka ar Nvidia videokarti
stabilākā un veiktspējīgākā izvēle ir CUDA, bet, ja mērķis ir atdalīties no
CUDA ekosistēmas, tad, izmantojot AMD izstrādāto CUDA savietojamo platformu un
migrācijas rīkus, ar 2\% - 3\% ātrdarbības zudumiem ir iespējams lietot ROCm HIP
platformu.

Ja mērķis ir vēl lielāks atbalstīto aparatūras paātrinātāju klāsts, kas
neaprobežojas ar Nvidia vai AMD videokartēm, bet arī ar citu firmu, tad ar
7\% - 30\% ātrdarbības zudumiem var izmantot OpenCL. Palielinātais iekārtu
atbalsts iekļauj arī citus paātrinātājus, kā FPGA (\textit{Field Programmable
Gate Array}), ciparsignālu un māklsīgā intelekta procesorus, kā arī pašu
centrālo procesoru.

Līdz ar to OpenCL ir kompromiss starp ātrdarbību un iekārtu atbalstu. Vienu un
to pašu GPGPU kodola kodu programmas  izpildes laikā ir iespējams kompilēt un
darbināt ar dažādiem aparatūras paātrinātājiem, bet, ja tādi  attiecīgajam
datoram nav pieejami, tad kodolu var darbināt ar centrālo procesoru.
