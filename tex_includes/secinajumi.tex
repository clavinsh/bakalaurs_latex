\begin{center}
    \chapter{Secinājumi}
\end{center}

Bakalaura darba mērķis bija teorētiski salīdzināt CUDA, ROCm un OpenCL
programmēšanas modeļus, valodu un piedāvāto bibliotēku funkcijas, bet lielāks
uzsvars likts uz  praktisko - izmērīt un analizēt platformu veiktspēju pēc
dažādiem to parametriem, nonākt pie secinājuma par to, kādiem darbiem kura
platforma paredzēta.

Ņemot vērā, ka visas trīs ir domātas paralēlas izpildes iekārtām, tad augstā
abstrakcijas līmenī tās ir diezgan līdzīgas - programmas CPU puse pārvalda par
iekārtas atmiņas iedalīšanu, datu izsūtīšanu un saņemšanu, kodolu darbināšanu.
Atšķirības, protams, ir detaļās - programmēšanas modelis, valodu specifikācija,
programmēšanas pieejamās C/C++ saskarņu funkcijas un to nianses, atbalstītās
iekārtas.

Praktiskai rīku salīdzināšanai, lai iegūtu uzticamus skaitliskus mērījumus, 
bija nepieciešamība iepazīties ar labās prakses etalonuzdevumu izveidi.
Rezultātā izveidotais ietvars tika lietots etalonuzdevumu skriptu, tā arī pašu
mērāmo programmu izstrādei.

Pēc iegūtajiem etalonuzdevumu datiem secināts, ka uz Nvidia videokartes
stabilākā un veiktspējīgākā izvēle bez pārsteigumiem ir CUDA, bet, ja mērķis ir
migrēties prom no CUDA ekosistēmas, tad, izmantojot AMD izstrādāto CUDA
savietojamo platformu un migrācijas rīkus, ar 2\%-\%3 ātrdarbības zudumiem ir
iespējams lietot ROCM HIP platformu.

Ja mērķis ir vēl lielāks atbalstīto aparatūras paātrinātāju klāsts, kas
neaprobežojas ar Nvidia vai AMD videokartēm, bet arī citu firmu, tad ar
7\%-30\% ātrdarbības zudumiem var izmantot OpenCL. Palielinātais iekārtu
atbalsts iekļauj arī citus paātrinātājus kā FPGA (\textit{Field Programmable
Gate Array}, ciparsignālu un māklsīgā intelekta procesorus, kā arī pašu
centrālo procesoru.

Līdz ar to OpenCL ir kompromiss starp ātrdarbību un iekārtu atbalstu. Vienu un
to pašu GPGPU kodola kodu ir iespējams programmas izpildes laikā kompilēt un
darbināt uz dažādiem aparatūras paātrinātājiemm, kā arī pat uz tā paša centrālā
procesora, ja tādi paātrinātāji attiecīgajam datoram nav pieejami.

