\begin{center}
    \chapter{Platformu GPGPU programmēšanas modeļi un lietojumprogrammu
    saskarne}
\end{center}


kursa darbā jau tika apskatītas galvenās atšķīrības un kopējās lietas starp
CUDA un ROCm

profilēšanas iespējas, gan no ārpuses (nvidia nsights, rocm sys profiler) un 
no api iekšpuses (eventi..)


hipificēšana nestrādā/strādā šādos gadījumos...
te varētu konkrētu piemēru kurš uz CUDA kompilējas, bet ne uz HIP 

visu platformu piedāvātie atmiņu tipi (pārsvarā jau ir diezgan līdzīgi)


